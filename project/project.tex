\documentclass{article}
\usepackage{fontspec}

% Used to embed Sage code in latex
\usepackage{sagetex}

% Physics Environment
\usepackage{physics}        % Physics symbols

% Math Environment
\usepackage{euler}        % Euler font
\usepackage{amsmath}      % Math macros
\usepackage{amssymb}      % Math symbols
\usepackage{unicode-math} % Unicode support

% Physics Environment
%\usepackage{physics}        % Physics symbols

\usepackage[makeroom]{cancel} % Used to cancel terms in algebraic equations
\usepackage{ulem} % Different underline environments
\usepackage{polynom} %Polynomial long division

% Typesetting Rules
\setlength\parindent{0em}
\setlength\parskip{0.618em}
\usepackage[a4paper,lmargin=1in,rmargin=1in,tmargin=1in,bmargin=1in]{geometry}
\setmainfont[Mapping=tex-text]{Helvetica Neue LT Std 45 Light}

% Common Macros
\newcommand\N{\mathbb{N}}
\newcommand\Z{\mathbb{Z}}
\newcommand\R{\mathbb{R}}
\newcommand\C{\mathbb{C}}
\newcommand\A{\mathbb{A}}
\def\res{\mathop{\text{Res}}\limits}

% Color
\usepackage[dvipsnames]{xcolor}
\usepackage{pagecolor}
\definecolor{DeepCyan}{HTML}{006969}
\definecolor{DeepRed}{HTML}{690000}
\pagecolor{DeepCyan}
\color{white}
\usepackage{abstract}
\begin{document}

\begin{center}
  Interactive Visualization of the Riemann Sphere

  MATH 165B - Final Project

  Ricardo J. Acuña

  \today{}
\end{center}\vspace{1.618em}
\begin{abstract}
  While algebraic manipulation of complex valued functions yield
  direct formulas for geometric objects, it's often difficult to see
  what object an equation represents. With
  the advent of computers we're able to directly visualize complicated
  equations. In this paper, we'll provide methods to graph such
  complex valued equations both in the Complex Plane and in the
  Riemann Sphere.
  \end{abstract}

\section{Introduction}

A function $f: \C^2\circlearrowleft$ is initially hard to grasp
geometrically. If we
consider the analogue of a function $g:R\circlearrowleft$ we can
consider it's graph $\{(x,y)\in \R^2|\, y = g(x)\}$ which is a $2$D
object which we can grasp as a picture in the plane. However, the
graph of $f = u+iv$ is $\{(z,w)\in \C^2|\, w = g(z)\} \simeq
\{(x,y,w_1,w_2)\in \R^4|\, w_1 = u(x,y),\text{ and } w_2 = v(x,y)\}$
which is $4$D, more dimensions than the ones we can experience.

So what we usually do is break up the domain and the codomain and
graph them separately. And then we consider the function $f$ as an
action that transforms the $x,y$-plane into the $u,v$-plane.

However, initially there is no necessary geometric meaning of the set
$\C$. But, we can give it such a meaning. We can see there is canonical isomorphism $\phi:\C \xrightarrow{\sim} \R^2$ between the set of complex
numbers $\C$, and the set of pairs of real numbers $\R^2$, which gives
us a coordinate system for the Euclidean plane. If we let
$z\in \C,$ then the isomorphism $\phi$ is given by,
\[\phi(z) = (\Re(z),\Im(z))\]

There are infinitely many ways we can embed $\R^2$ into
$\R^3$. However if we let,
$(x,y) \in \R^2$, it's natural to identify the plane with the
inclusion map $\iota:\R^2\hookrightarrow \R^3$
\[\iota(x,y) = (x,y,0)\]


The Riemann Sphere is an isomorphism between the Extended Complex
Plane $\C\cup \{\infty\}$, and the points the a unit sphere $S^2$ centered at
the origin. The isomorphism given by a formula called stereographic
projection.

In Visual Complex Analysis page 146 formula (20), Tristan Needham
derives that formula, if a point on $S^2$ is given by the
cartesian coordinates $(X,Y,Z)\in \R^3$, and a point in the extended
Complex Plane$z = x+iy\in \C\cup \{\infty\}$ the stereographic
projection is,
\[(X,Y,Z)\mapsto \frac{X}{1-Z}+i\frac{Y}{1-Z}\]

And the inverse stereographic projection is given by,
\[x+iy \mapsto \left( \frac{2x}{1+x^2+y^2},\frac{2y}{1+x^2+y^2},
    \frac{-1+x^2+y^2}{1+x^2+y^2}\right)\]

The derivation uses the plane $z=0$, however notice that the
formula for the inverse stereographic projection is independent of the
lowercase $z$. That means, we can embed the plane into $\R^3$ however
we like, we do so by the mapping $\iota_{-1}$ where,
\[\iota_{-1}: \C\hookrightarrow\R^3;
  x+iy\mapsto (x,y,-1)\]

The reason we want to embed this way is because, while we could make a
$3$D rendering of $S^2$, and a $2$D rendering of $\C$ side by
side. It's much nicer to look at them together as a $3$D graph. And in the derivation
notice, that the plane $z = 0$, slices $S^2$ in the middle, so much of
the detail of the stereographic projection would be muddled.


\section{Method}

At first I didn't know how to approach the problem, so I did the
na\"ive thing and made a couple of spheres, together with a plane.

The first one was made using clojurescript, and a rendering library
called quil,

\begin{verbatim}
(ns sphere.core
  (:require [quil.core :as q :include-macros true]
            [quil.middleware :as m]))

(def width (. js/window -innerWidth))
(def height (. js/window -innerHeight))

(defn draw [state]
  (q/background 255)
  (q/lights)
  (q/fill 230 110 110)
  (q/with-translation [[(/ width 2) (/ height 2) 0]]
    (q/fill 255 0 0)
    (q/curve 0 0 0 0 0 0 0 0 0 (* 50 width) 0 0)
    (q/curve 0 0 0 0 0 0 0 0 0 (- (* 50 width)) 0 0)
    (q/fill 0 255 0)
    (q/curve 0 0 0 0 0 0 0 0 0 0 (* 50 width) 0)
    (q/curve 0 0 0 0 0 0 0 0 0 0 (- (* 50 width)) 0)
    (q/fill 0 0 255)
    (q/curve 0 0 0 0 0 0 0 0 0 0 0 (* 50 width))
    (q/curve 0 0 0 0 0 0 0 0 0 0 0 (- (* 50 width)))
    (q/fill 110 110 230)
    (q/with-translation [[0 100 0]]
      (q/rotate-x (/ 3.1415 2))
      (q/plane (* 50 width) (* 50 width)))
    (q/fill (q/color 110 230 110 120))
    (q/with-translation [[0 0 0]]
      (q/sphere 100)
      (q/with-translation [[0 -101 0]]
        (q/sphere 1)))))

(q/defsketch sphere
  :host "sphere"
  :draw draw
  :renderer :p3d
  :middleware [m/fun-mode m/navigation-3d]
  :size [width height])
\end{verbatim}
\begin{center}
\includegraphics[width = 0.4\textwidth]{quil-sphere.png}
\end{center}

However, that package didn't allow enough control on the movement of
the sphere. Next, I used the same computer language, but now with a
library called threeagent. I was able to add controls such that the
position of the sphere changed when you clicked the up and down
arrows.
\begin{verbatim}
(ns sphere.app
  (:require ["three" :as three]
            [threeagent.alpha.core :as th]
            [reagent.core :as r]))

(def sphere-velocity 5.0)

(defonce state (th/atom {:sphere {:right  {:height 3.0
                                          :position [0 0 0]}}}))
(def world-scale 1.0)

(defn sphere [side]
  (let [[x-pos y-pos z-pos] @(th/cursor state [:sphere side :position])
        height @(th/cursor state [:sphere side :height])]
    [:object
     [:point-light {:intensity 1.8
                    :position [x-pos (+ y-pos 2.3) z-pos]}]
     [:sphere {:radius 2
               :position [x-pos y-pos z-pos]
               :width-segments 32
               :height-segments 32
               :material {:color "green"
                          :transparency 0.5
                          :transparent true}}]]))

(defn root []
  [:object
   [:hemisphere-light {:position [0 10 10]
                       :intensity 0.4}]
   [:object {:position [0 0 -10]
             :scale [world-scale world-scale 1]}
    [:box {:position [0 -2 0]
           :scale [20 0.1 20]
           :material {:color "blue"
                      :transparency 0.9
                      :transparent true}}]
    [sphere :right]]])

(defn- update-sphere! [delta-time]
  (doseq [p [:right]]
    (let [[px py pz] (get-in @state [:sphere p :position])
          velocity (get-in @state [:sphere p :velocity])
          new-py (+ py (* delta-time velocity))]
      (swap! state assoc-in [:sphere p :position] [px new-py pz]))))

(defn- tick [delta-time]
  (update-sphere! delta-time))

(defn on-keydown [evt]
  (case (.-code evt)
    "ArrowUp" (do
                (.preventDefault evt)
                (swap! state assoc-in [:sphere :right :velocity] sphere-velocity))
    "ArrowDown" (do
                  (.preventDefault evt)
                  (swap! state assoc-in [:sphere :right :velocity] (- sphere-velocity)))
    nil))

(defn- on-keyup [evt]
  (case (.-code evt)
    "ArrowUp" (swap! state assoc-in [:sphere :right :velocity] 0)
    "ArrowDown" (swap! state assoc-in [:sphere :right :velocity] 0)
    nil))

(defn init []
  (th/render [root]
             (.getElementById js/document "root")
             {:on-before-render tick})
  (.addEventListener js/window "keydown" on-keydown)
  (.addEventListener js/window "keyup" on-keyup))

(defn ^:dev/after-load reload []
  (th/render [root]
             (.getElementById js/document "root")
             {:on-before-render tick}))
\end{verbatim}

A light was placed atop the sphere in the hopes I could shine it
through a texture on the sphere and project it's shadow onto the
plane, such that as I moved the sphere, the corresponding shadow on
the plane changed. The result is as follows,

\begin{minipage}{0.5\textwidth}
  \includegraphics[width = 0.8\textwidth]{threeagent1.png}
\end{minipage}
\begin{minipage}{0.5\textwidth}
  \includegraphics[width = 0.86\textwidth]{threeagent2.png}.
\end{minipage}

However, there were problems with the library's control of the
transparency. Next, I decided to switch to Javascript and use the
library three.js directly, as three.js was the backend for
threeagent. What I lost was the ability to control the sphere, but I
gained control over the transparency of the sphere.

\begin{verbatim}
import * as THREE from 'https://threejsfundamentals.org/threejs/resources/threejs/r115/build/three.module.js';
import {OrbitControls} from 'https://threejsfundamentals.org/threejs/resources/threejs/r115/examples/jsm/controls/OrbitControls.js';

function main() {
  const canvas = document.querySelector('#c');
  const renderer = new THREE.WebGLRenderer({canvas});

  const fov = 75;
  const aspect = 2;  // the canvas default
  const near = 0.1;
  const far = 25;
  const camera = new THREE.PerspectiveCamera(fov, aspect, near, far);
  camera.position.z = 4;

  const controls = new OrbitControls(camera, canvas);
  controls.target.set(0, 0, 0);
  controls.update();

  const scene = new THREE.Scene();
  scene.background = new THREE.Color('black');

  function addLight(...pos) {
    const color = 0xFFFFFF;
    const intensity = 1;
    const light = new THREE.DirectionalLight(color, intensity);
    light.position.set(...pos);
    scene.add(light);
  }
	// addLight(-1, 2, 4);
  // addLight( 1, -1, -2);

  const radius = 1;
  const widthSegments = 32;
	const heightSegments = 32;
  const sphereGeometry = new THREE.SphereBufferGeometry(radius, widthSegments,heightSegments);
	const planeGeometry = new THREE.PlaneGeometry( 5, 20, 32 );

	function hsl(h, s, l) {
    return (new THREE.Color()).setHSL(h, s, l);
  }

  function makeSphere(geometry, color, x, y, z) {
    const material = new THREE.MeshPhysicalMaterial({
      color: color,
			metalness: 0,
			alphaTest: 0.5,
			roughness: 0,
			depthWrite: false,
			transparency: 0.5,
      opacity: 1,
      transparent: true,
    });

    const sphere = new THREE.Mesh(geometry, material);
    scene.add(sphere);
		addLight(x,y+1,z)
    sphere.position.set(x, y, z);

    return sphere;
  }


	function makePlane(geometry, color, x, y, z) {
		const material = new THREE.MeshBasicMaterial( {color: color, side: THREE.DoubleSide} );

		const plane = new THREE.Mesh(geometry, material );
		scene.add( plane );


		plane.position.set(x, y-1, z);
		plane.rotateX(3.14/2)

		return plane;
	}

	makeSphere(sphereGeometry,hsl(5/8,1,.5),0,0,0)
	makePlane(planeGeometry,"white",0,0,0)

  function resizeRendererToDisplaySize(renderer) {
    const canvas = renderer.domElement;
    const width = canvas.clientWidth;
    const height = canvas.clientHeight;
    const needResize = canvas.width !== width || canvas.height !== height;
    if (needResize) {
      renderer.setSize(width, height, false);
    }
    return needResize;
  }

  let renderRequested = false;

  function render() {
    renderRequested = undefined;

    if (resizeRendererToDisplaySize(renderer)) {
      const canvas = renderer.domElement;
      camera.aspect = canvas.clientWidth / canvas.clientHeight;
      camera.updateProjectionMatrix();
    }

    renderer.render(scene, camera);
  }
  render();

  function requestRenderIfNotRequested() {
    if (!renderRequested) {
      renderRequested = true;
      requestAnimationFrame(render);
    }
  }

  controls.addEventListener('change', requestRenderIfNotRequested);
  window.addEventListener('resize', requestRenderIfNotRequested);
}

main();
\end{verbatim}
As you can see its quite a bit more verbose, the resulting sphere is
this,
\begin{center}
  \includegraphics[width=0.3\textwidth]{threejs-sphere.png}
  \end{center}
Notice, it doesn't cast shadows. Actually casting shadows is a rather
expensive in terms of computation. Its done by a technique called ray
casting.

In the paper Moeb\"ius Transformations revealed Douglas
N. Arnold and Jonathan Rogness reveal that through that technique they
made the animation found in the popular YouTube video of the same
name. This project started as an attempt to reproduce the visual
effects found in that video. Since, the previous attempts didn't yield
satisfactory graphics, I decided to study their methods. However, in
the paper they don't give any code whatsoever, they just say it was
easy to do with the package POVRayRender with some help of
Mathematica.

In any case I set out to reproduce their methods, I found that an
extension of Mathematica allows us to output POVrayrender images out
of Mathematica graphs. While it was rather technically difficult to
install, because there were some issues with Linux compatibility, I
managed to make the following sphere. First by making the graphic of a
sphere with a plane in Mathematica, and then modifying the source code
of some intermediate temporary POVRayrender file
to change the transparency of the material and the pattern in the
plane. The code is as follows,

\begin{verbatim}
#include "colors.inc"

global_settings {
  max_trace_level 5
  assumed_gamma 1.0
  radiosity {
    pretrace_start 0.08
    pretrace_end   0.01
    count 35
    nearest_count 5
    error_bound 1.8
    recursion_limit 2
    low_error_factor .5
    gray_threshold 0.0
    minimum_reuse 0.015
    brightness 1
    adc_bailout 0.01/2
  }
}

#default {
  texture {
    pigment {rgb 1}
    finish {
      ambient 0.0
      diffuse 0.6
      specular 0.6 roughness 0.001
      reflection { 0.0 1.0 fresnel on }
      conserve_energy
    }
  }
}

light_source {
    <0, 0, 1>
    color White
    cylinder
    radius 15
    falloff 20
    tightness 10
    point_at <0, 0, 0>
}
light_source {
    <0, 1, 0>
    color White
    cylinder
    radius 15
    falloff 20
    tightness 10
    point_at <0, 0, 0>
}

light_source {
    <1, 0, 0>
    color White
    cylinder
    radius 15
    falloff 20
    tightness 10
    point_at <0, 0, 0>
}

background { rgb <0,.25,.5> }

camera {
	location <3.9000000000000004, -7.199999999999999, 5.5>
	direction <-3.9000000000000004, 7.199999999999999, -6.>
	up <0, 0, 4.242640687119286>
	right <5.64271211386865, 0, 0>
	sky <0., 0., 1.>
	look_at <0., 0., -0.5>
}

sphere {
	<0., 0., 0.>,
	1

	pigment  { rgbf <1,1,1,0.95> }
	finish   { phong 0.9 phong_size 40
		 // A highlight
	           reflection 0.2
		 // Glass reflects a bit
		 }
	interior { ior 1.5}
		 // Glass refraction

}

plane { z,-1
	texture{
		pigment{ checker rgb<1,1,1> rgb<0,0,0>}
		scale 0.25
		}
	}
\end{verbatim}
\begin{center}
  \includegraphics[width = 0.5\textwidth]{pov-sphere.png}
  \end{center}
Which is really nice, notice that due to the transparency, the upper
hemisphere of the sphere is a conformal mapping. Which means that if
you look at the image of the plane in the sphere, the lines of the
checkered pattern remain locally at right angles.

Some further research lead me to the code by Xah Le who created a
Mathematica package for M\"obius transformations, that included a way
to use Mathematica to make a
stereographic projection. He sells the library that he used to make
his examples for \$15, I bought it. However, he took a whole day to
email me the code. So, by then I had figured out how to do it on my
own in the Sage programming environment. This is my first prototype

\begin{verbatim}
from sage.plot.plot3d.shapes2 import line3d
from sage.plot.plot3d.shapes import Sphere

import numpy as np
interval = np.arange(-2,2.25,0.25)
g = Sphere(1, color = "white", alpha = 0.1)
def isp (p):
    x,y,z = p
    r = 1 + x^2 +y^2
    return (2*x/r,2*y/r,(r-2)/r)

def iz (x,y):
    return (x,y,-1)

for j in interval:
    g += line3d([iz(i,j) for i in interval],color = "blue")
    g += line3d([isp(iz(i,j)) for i in interval], color = "blue")
    g += line3d([iz(j,i) for i in interval], color = "red")
    g += line3d([isp(iz(j,i)) for i in interval], color ="red")

g.show(aspect_ratio=[1,1,1])
\end{verbatim}
\begin{center}
  \includegraphics[width = 0.5\textwidth]{prototype.png}
  \end{center}

Notice, the function iz is the embedding $\iota_{-1}$, and isp is the
inverse stereographic projection. The prototype takes lines on the
plane to lines on $S^3$.

However, I wanted a little more control on the transparency, so I made
the following Sphere.

\begin{verbatim}
from sage.plot.plot3d.shapes2 import polygon3d
import numpy as np

def isp (p):
    x,y,z = p
    r = 1 + x^2 +y^2
    return (2*x/r,2*y/r,(r-2)/r)

def ix_3 (z):
    x,y = z
    return (x,y,-1)

def make_square (pt,l,c):
    x,y = pt
    s = [(x,y),(x+l,y),(x+l,y+l),(x,y+l)]
    return polygon3d([ix_3(z) for z in s],color = c)

def make_sector (pt,l,c):
    x,y = pt
    s = [(x,y),(x+l,y),(x+l,y+l),(x,y+l)]
    return polygon3d([isp(ix_3(z)) for z in s],color = c,alpha = 0.75)

def make_instance(f):
    instance = Graphics()
    step = 0.25
    interval = np.arange(-2,2+step,step)
    for j in interval:
        for i in interval:
            if (i*4)%2 == 0 and (j*4)%2 != 0 :
                instance += f((0+i,j),step,"red");
            elif (i*4)%2 == 0 and (j*4)%2 == 0 :
                instance += f((0+i,j),step,"green");
            elif (i*4)%2 != 0 and (j*4)%2 != 0 :
                instance += f((0+i,j),step,"yellow");
            else:
                instance += f((0+i,j),step,"blue");
    return instance

g = make_instance(make_square) + make_instance(make_sector)
g.show(viewer='threejs',aspect_ratio=[1,1,1],axes=False,frame=False,online=True)
\end{verbatim}

The function ix\_3 is the same as iz but it acts on points in $\R^2$,
instead of having arity 2 with arguments x, and y, it has only the
single argument z which must be pair.

The function make\_square makes a square, but also it can be extended
to a constructor that maps the image of a square under a complex
function by applying a map before doing the inclusion on the edges.

The function make\_sector makes the image of a square under the inverse
stereographic projection, but again it can be extended to the image of
a complex map on $S^2$ by applying a map before doing the
inclusion.

The key step for the conclusion of this project is in the last line of
the code. The keyword online=True, gives the compiler the instruction
to include the sources of all the required dependencies in the HTML
file of the Sage Notebook where this runs. On the Sage Notebook, I
downloaded a HTML version of the .ipynb source file. Which then I
managed to modify by deleting all the elements that weren't the
visualization.

So, I got a standalone $3$D visualization that can be
zoomed in and zoomed out and rotated.

\begin{center}
  \includegraphics[width = 0.5\textwidth]{final-sphere.png}
  \end{center}

And actually the visualization can be found in the url, http://sphere.drakezhard.org

\section{Conclusions --- Future Work}

While the project was quite ambitious, it lent itself quite well to
computation. I've shown in the previous section that the code is
easily extensible to arbitrary complex mappings.

However, a major
thing about the YouTube video M\"obius Transformations Revealed was
that the shadow of a pattern on $S^2$ corresponds to a transformation
of the plane as the $S^2$ gets translated and rotated.

So a natural direction of future work, would be to combine the ideas
about tiling the plane and inverse projecting onto the sphere, and the
code where I managed to make the sphere move. The challenge would be
to rework the code into the other language.

\section{References}

Needham, T.  Visual Complex Analysis. Oxford University Press, 1997.

Arnold, D; Rogness, J.  Möbius Transformations Revealed. Notices of
The AMS, Vol. 55, Num. 10.

Lee, X. Stereographic Projection. Online, 2016.

\hspace{3 em} URL http://xahlee.info/MathGraphicsGallery\_dir/sphere\_projection/index.html
\end{document}
%%% Local Variables:
%%% mode: latex
%%% TeX-master: t
%%% End:
