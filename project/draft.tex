\documentclass{article}
\usepackage{fontspec}

% Used to embed Sage code in latex
\usepackage{sagetex}

% Physics Environment
\usepackage{physics}        % Physics symbols

% Math Environment
\usepackage{euler}        % Euler font
\usepackage{amsmath}      % Math macros
\usepackage{amssymb}      % Math symbols
\usepackage{unicode-math} % Unicode support

% Physics Environment
\usepackage{physics}        % Physics symbols

\usepackage[makeroom]{cancel} % Used to cancel terms in algebraic equations
\usepackage{ulem} % Different underline environments
\usepackage{polynom} %Polynomial long division

% Typesetting Rules
\setlength\parindent{0em}
\setlength\parskip{0.618em}
\usepackage[a4paper,lmargin=1in,rmargin=1in,tmargin=1in,bmargin=1in]{geometry}
\setmainfont[Mapping=tex-text]{Helvetica Neue LT Std 45 Light}

% Common Macros
\newcommand\N{\mathbb{N}}
\newcommand\Z{\mathbb{Z}}
\newcommand\R{\mathbb{R}}
\newcommand\C{\mathbb{C}}
\newcommand\A{\mathbb{A}}
\def\res{\mathop{\text{Res}}\limits}

% Color
\usepackage[dvipsnames]{xcolor}
\usepackage{pagecolor}
\definecolor{DeepCyan}{HTML}{006969}
\definecolor{DeepRed}{HTML}{690000}
\pagecolor{DeepCyan}
\color{white}

\begin{document}

\begin{center}
  MATH 165B - Introduction to

  Complex Variables
\end{center}\vspace{1.618em}

Names:
\begin{enumerate}
\item Ricardo Jaime Acuna
  \end{enumerate}
Topic: An interactive exploration of complex valued mappings through
visualizations of the Riemann Sphere

Sub Topics:
\begin{enumerate}
  \item Stereographic Projection.
  \item M\"obius Transformations.
  \item Conformal Mappings.
  \item Computer Graphics.
\end{enumerate}

Problem:

I want to make a web application that allows one to visualize M\"obius
transformations, and Conformal Mappings by Stereographic Projection
of the Riemann Sphere onto the Complex Plane.

Following the video Möbius Transformations Revealed  by Douglas Arnold
and Jonathan Rogness, I want to shine a light atop of the Riemann
Sphere that will cast rays through the surface of the sphere onto the
Complex Plane.

The semi-transparent grid of the sphere will then be
projected onto the plane, the shadow of the sphere on the plane will
transform
with the sphere as the sphere gets rotated and inverted.

Then we
will be able to apply a conformal map to the grid and see the image of
the mapping both on the plane and on the sphere, and see what happens
under m\"obius transformation to the image of the conformal map.

Another possible turn of this project is to see conformal mappings as
wrappings of the plane onto surfaces. Like when you
see yourself in a fun house mirror.

Computations:

I will use the computer language Clojurescript, and the
rendering library Reagent, together with a wrapper of three.js called
threeagent that allows me to make WebGl powered 3D applications that
run in web browsers. On the server side, I will use static web hosting through Amazon Web
Services buckets.

You will find the code at: https://www.github.com/rjacuna/sphere

You will find the application at: http://www.drakezhard.org

Bibliography:

Trott, M. The Mathematica GuideBook for Programming. New York:
Springer-Verlag, 2004.

Kythe, P. K. Computational Conformal Mapping. Boston, MA: Birkhäuser, 1998.

Needham, T. Visual Complex Analysis. Clarendon Press, 1998

Communications and Work Group Plan:

1. I will constantly remind my right
hemisphere to get this done. I'll leave the homework to the left
hemisphere for the remainder of the quarter.

2. The files will be uploaded to github.com/rjacuna, and
drakezhard.org as soon as any progress is made.

3. Math will be at the center of everything. In fact If I don't use
math in this project I won't be able to get anything done.

\end{document}
%%% Local Variables:
%%% mode: latex
%%% TeX-master: t
%%% End:
