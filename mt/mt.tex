\documentclass{article}
\usepackage{fontspec}

% Used to embed Sage code in latex
\usepackage{sagetex}

% Math Environment
\usepackage{euler}        % Euler font
\usepackage{amsmath}      % Math macros
\usepackage{amssymb}      % Math symbols
\usepackage{unicode-math} % Unicode support


\usepackage[makeroom]{cancel} % Used to cancel terms in algebraic equations
\usepackage{ulem} % Different underline environments
\usepackage{polynom} %Polynomial long division

% Typesetting Rules
\setlength\parindent{0em}
\setlength\parskip{0.618em}
\usepackage[a4paper,lmargin=1in,rmargin=1in,tmargin=1in,bmargin=1in]{geometry}
\setmainfont[Mapping=tex-text]{Helvetica Neue LT Std 45 Light}

% Common Macros
\newcommand\N{\mathbb{N}}
\newcommand\Z{\mathbb{Z}}
\newcommand\R{\mathbb{R}}
\newcommand\C{\mathbb{C}}
\newcommand\A{\mathbb{A}}
\def\res{\mathop{\text{Res}}\limits}

% Color
\usepackage[dvipsnames]{xcolor}
\usepackage{pagecolor}
\definecolor{DeepCyan}{HTML}{006969}
\definecolor{DeepRed}{HTML}{690000}
\pagecolor{DeepCyan}
\color{Goldenrod}

\begin{document}

\begin{center}
  165B --- Midterm Exam

  RJ Acuña

  (862079740)
\end{center}\vspace{1.618em}

\subsubsection*{Problem 1} In each case, write the principal part of the function at its isolated singular point and determine
whether that point is a pole, a removable singular point, or an
essential singular point:

(a) $\exp\left(\frac{1}{z^2}\right)$

\uwave{slu.}

$\exp\left(\frac{1}{z^2}\right)$ has an isolated singular point at $0$, we will
see why from it's series expansion about $0$.

We know that, $\exp(z) = \sum_{n=0}^\infty \frac{z^n}{n!}$. Thus,

\[\exp\left(\frac{1}{z^2}\right) = \sum_{n=0}^\infty \frac{\left(
    \frac{1}{z^2}\right)^n}{n!} = \sum_{n=0}^\infty \frac{1}{n!z^{2n}}
=1 +\sum_{n=1}^\infty \frac{1}{n!z^{2n}}\]

The principal part is then, \[\sum_{n=1}^\infty \frac{1}{n!z^{2n}}\]

Since the principal part has infinitely many non-zero coefficients. It
follows that, $0$ is an essential singularity of
$\exp\left(\frac{1}{z^2}\right)\quad \blacklozenge$

(b) $\frac{z^3}{1-z}$

We can see that $\frac{z^3}{1-z}$ is not analytic at $z=1$. Then we
perform polynomial long division,

\begin{minipage}{.35\textwidth}
\polyset{vars=z}
\polylongdiv[style = A]{z^3}{1-z}
\end{minipage}
\begin{minipage}{.65\textwidth}
  And we can conclude that,
  \[\frac{z^3}{1-z} = -z^2-z-1 + \frac{1}{1-z}\]
  So the principal part is, \[\frac{1}{1-z}\]

  So $1$ is a simple pole $\quad \blacklozenge$
\end{minipage}

(c) $\frac{\sin 2z}{z}$


\uwave{slu.}

\[\frac{\sin 2z}{z} = \frac{1}{z}\sum_{n=0}^\infty
\frac{(-1)^n(2z)^{2n+1}}{(2n+1)!} = \sum_{n=0}^\infty
\frac{(-1)^n2^{2n+1}z^{2n}}{(2n+1)!}\]

Since $\sin 2z$ is analytic, but $\frac{1}{z}$ is not analytic at
$0$. $0$ is the isolated singular point of $\frac{\sin 2z}{z}$, then
we compute the following expansion.

\[\frac{\sin 2z}{z} = \frac{1}{z}\sum_{n=0}^\infty
\frac{(-1)^n(2z)^{2n+1}}{(2n+1)!} = \sum_{n=0}^\infty
\frac{(-1)^n2^{2n+1}z^{2n}}{(2n+1)!}\]

Thus, the principal part is $0$, therefore $0$ is a removable
singularity of $\frac{\sin 2z}{z}\quad \lozenge$
\newpage
(d) $\frac{\cos z -1}{z^2}$

\uwave{slu.}

$\cos z -1$ is analytic, but but $\frac{1}{z^2}$ is not analytic at
$0$. $0$ is the isolated singular point of $\frac{\cos z -1}{z^2}$, then
we compute the following expansion.

\[\frac{\cos z -1}{z} = \frac{1}{z^2}\left( \sum_{n=0}^\infty
\frac{(-1)^nz^{2n}}{(2n)!} -1\right) = \frac{1}{z^2}\left( 1+ \sum_{n=1}^\infty
\frac{(-1)^nz^{2n}}{(2n)!} -1\right) = \frac{1}{z^2}\left(\sum_{n=1}^\infty
\frac{(-1)^nz^{2n}}{(2n)!}\right) = \sum_{n=1}^\infty
\frac{(-1)^nz^{2n-2}}{(2n)!}\]

Since there are no negative powers, the principal part is $0$, therefore $0$ is a removable
singularity of $\frac{\cos z -1}{z^2}\quad \lozenge$

(e) $\frac{1}{(1-z)^3}$

\uwave{slu.}
Since it's already in it's Laurent series representation we can see
that $1$ is a pole of order $3$ of $\frac{1}{(1-z)^3}\quad \lozenge$

\subsubsection*{Problem 2} Find

(a) The residue of $f_1(z) = \frac{\pi}{z-z^2}$ at $z=0$

\uwave{slu. } $f_1(z) = \frac{\pi}{z(1-z)} = \frac{\frac{\pi}{1-z}}{z}
\implies \res_{z=0}f_1(z) = \frac{pi}{1-0} = \pi\quad \lozenge$

(b) The residue of $f_2(z) = z\cos\left( \frac{1}{z} \right)$ at $z=0$

\uwave{slu.}
\[f_2(z) = z\cos\left( \frac{1}{z} \right) = z\sum_{n=0}^{\infty}
  \frac{(-1)^n\left( \frac{1}{z} \right)^{2n}}{(2n)!}  = z\sum_{n=0}^{\infty}
  \frac{(-1)^n}{(2n)!z^{2n}}= \sum_{n=0}^{\infty}
  \frac{(-1)^n}{(2n)!z^{2n-1}}\]
\[\implies f_2(z) =\frac{(-1)^0}{0! z^{-1}} + \frac{(-1)^1}{2! z^{2-1}}+ \sum_{n=}^{\infty}
  \frac{(-1)^n}{(2n)!z^{2n-1}} = z - \frac{1}{2z}+ \sum_{n=}^{\infty}
  \frac{(-1)^n}{(2n)!z^{2n-1}}\]
\[\implies \res_{z=0}f_2(z) = -\frac{1}{2} \quad \blacklozenge\]

(c) The residue of $f_3 (z) = \frac{z−\sin z }{2z}$ at $z = 0$

\uwave{slu.} $f_3(z) = \frac{z-\sin z}{2z} = \frac{\frac{z-\sin
    z}{2}}{z}$, since $\frac{z-\sin z}{2}$ is analytic at $0\implies
\res_{z=0}f_3(z) = \frac{0-\sin 0}{2}= 0\quad \lozenge$

(d) A function $f_4$ with a simple pole at $z = 0$ such that the
residue of $f_4$ at $z = 0 $ is $ π$.

\uwave{slu.} $f_4(z) = \frac{\pi}{z}\quad \lozenge$

(d) A function $f_5$ with a pole  of order $3$ at $z = 0$ such that the
residue of $f_5$ at $z = 0 $ is $17$.

\uwave{slu.} $f_5(z) = \frac{17}{z} +\frac{27891}{z^3}\quad \lozenge$
\newpage

\newpage
\subsubsection*{Problem 3} Consider the integral \[\int_C
  \frac{2z^3+3}{(z+1)(z^2 +4)} dz\] taken counterclockwise around the
curve $C$

(a) Find the value of the integral when the curve $C$ is the circle
${\color{green}|z + 1| = 2}$

\uwave{slu.}

\[(z+1)(z^4+4) = 0\implies z+1 = 0 \text{ or } z^2+4 = 0\]
\[z+1 = 0 \implies z_0 = -1\]
\[z^4+4 = 0 \implies z^2 = -4 = -1\cdot 4 = 4 e^{i\left(  \pi+
      2k\pi\right)},\,\, k\in \Z \implies z =
  \sqrt{4}e^{i\left(  \frac{\pi}{2}+ k\pi \right)}= \pm 2i\]
\[z^2+4 = 0 \implies z_1 = 2i,z_2 = -2i\]
\[z_0 = -1 \implies |-1+1| = 0 <2 \implies z_0\in C\]
\[z_1 = 2i \implies |2i+1| = |1+2i| = \sqrt{5}\]\[ 4<5 \implies
  \sqrt{4} = 2< \sqrt{5}  \implies z_1\not\in C\]
\[z_2 = -2i \implies |-2i+1| = |1-2i| = \sqrt{5} > 2\implies z_2 \not\in C\]

Note, that finding all the zeros of $z^2+4$ gives us the following factorization,
\[z^4+4 = (z-2i)(z+2i) \]
\[\text{Then } f(z) = \frac{2z^3+3}{(z+1)(z^2 +4)} = \frac{2z^3+3}{(z+1)(z-2i)(z+2i)}\]
\begin{align*}\implies \res_{z=-1} f(z) &=
  \frac{2(-1)^3+3}{(-1 -2i)(-1+2i)}= \frac{1}{5}
  \end{align*}

So Cauchy's residue theorem gives us that,
 \begin{align*}\int_C
  \frac{2z^3+3}{(z+1)(z^2 +4)} dz &= 2\pi i\left( \frac{1}{5}
                                    \right) = \frac{2\pi }{5}i\end{align*}

(b) find the value of the integral when the curve $C$ is the circle
$|z|=4$

Since, $|\pm 2i| = 2 < 4$, and $|-1|=1 <4$, we need to
compute a few more residues.

\begin{sagesilent}
  var('z', domain = 'complex')
  f_1(z) = (2*z^3+3)/((z+1)*(z+2*I))
  r_1 = f_1(2*I)
  f_2(z) = (2*z^3+3)/((z+1)*(z-2*I))
  r_2 = f_2(-2*I)
\end{sagesilent}


\begin{align*}\implies \res_{z=2i} f(z) &=
  \frac{2(2i)^3+3}{(2i +1)(2i+2i)}= \frac{2(2i)^3+3}{(2i +1)2(2i)}=
                                          \frac{(2i)^2}{1+2i}
                                          -\frac{3i}{4(1+2i)}\\ &= \frac{-4(1-2i)}{5}
                                          -\frac{3i(1-2i)}{20}  =
                                          \frac{-16+32i-3i -6}{20}\\ &=
                                          \frac{-22+29i}{20}= \sage{r_1.full_simplify()}
\end{align*}

\begin{align*}\implies \res_{z=-2i} f(z)
  &=    \frac{2(-2i)^3+3}{(-2i +1)(-2i-2i)}
  = \frac{2(-2i)^3+3}{2(1-2i)(-2i)}
  =  \frac{(-2i)^2}{1-2i} +\frac{3}{2(1-2i)(-2i)}\\
    &=  \frac{(-2i)^2(1+2i)}{5} +\frac{3(1+2i)}{10(-2i)}
    =  \frac{-4(1+2i)}{5} -\frac{3i(1+2i)}{10(-2)}
    =  \frac{-16(1+2i)}{20} +\frac{3i(1+2i)}{20}\\
    &=  \frac{-16(1+2i) +3i(1+2i)}{20}
    =  \frac{(-16+3i)(1+2i)}{20}
    =  \frac{(-16-6)+(3 -32)i}{20}
    =  \frac{-22 -29i}{20}\\
                                                         &= \sage{r_2.full_simplify()}
  \end{align*}


\begin{sagesilent}
  var('z', domain = 'complex')
  f = (2*z^3+3)/((z+1)*(z^2+4))
  s = f.residue(z==-1)
  s +=f.residue(z== 2*I)
  s += f.residue(z== -2*I)
\end{sagesilent}
So Cauchy's residue theorem gives us that,
 \begin{align*}\int_C
  \frac{2z^3+3}{(z+1)(z^4 +4)} dz &= 2\pi i\left( \frac{1}{5} -\frac{11}{10} +\frac{29}{20}i
                                    -\frac{11}{10} -\frac{29}{20}i
                                    \right)\\ &=
                                                2\pi i\left(
\frac{2 - 22}{10}                                              \right)
                                               =
                                                -4\pi i \quad
                                                \blacksquare
 \end{align*}


 (c) Give a curve $C$ such that the value of the integral is $0$

 \uwave{slu. }

 Let $C$ be the circle $|z-20| = 1$

 Since, $|1-20| = 19 >1$, and $|\pm 2i - 20| = \sqrt{404} > \sqrt{400}
 = 20 >1$. So the interior of $C$, not $C$ contain any of the singular points of the
 integrand, thus the integrand is analytic in and on the circle of radius
 $1$ centered at $20$. Therefore, \[\int_C
  \frac{2z^3+3}{(z+1)(z^4 +4)} dz = 0\quad \lozenge\]

\subsubsection*{Problem 4} Problem 4: Show that the image of the right
half plane Re$(z) > \frac{1}{2}$ , under the mapping $w = \frac{1}{z}$, is the disk
$|w − 1| < 1 $.

\uwave{slu.}

Let $z$ in the right half plane Re$(z) > \frac{1}{2}$. Then $\exists
x,y\in \R: z= x +
iy: x >\frac{1}{2}.$

If $w = u+iv$, then
\[w = \frac{1}{z} =\frac{1}{x+iy}  = \frac{x-iy}{x^2+y^2} \implies u =
  \frac{x}{x^2+y^2}\text{ and } v = \frac{-y}{x^2+y^2} \]

\[w = \frac{1}{z} \implies z = \frac{1}{w} \implies x =
  \frac{u}{u^2+v^2}\text{ and } y = \frac{-v}{u^2+v^2}\]

The general equation for circles an lines in the plane is as follows,
\[A(x^2+y^2)+Bx + Cy + D = 0\]
And it gets mapped by $w$ to the $u,v$ plane into the following
equation,
\[D(u^2+v^2)+Bu - Cv + A = 0\]

First consider the boundary of the right half plane Re$(z) >
\frac{1}{2}$. Which is the line $x = \frac{1}{2}$

That is, $A = 0, B=1, C= 0, D=-\frac{1}{2}$ so the image is,
\[-\frac{1}{2}(u^2+v^2)+u - 0v + 0 = 0\implies u =
  \frac{1}{2}(u^2+v^2) \implies 2u =
  u^2+v^2\]
\[ \implies 0 =
  u^2-2u +v^2 \implies  1 =
  u^2-2u + 1 +v^2 \implies  1 =
  (u- 1)^2 +v^2\]

Which is the circle of radius $1$ centered at $1$.

Now, if $z$ is in the right half plane Re$(z) >
\frac{1}{2}$. $\forall z:$ Re$ (z) > \frac{1}{2}$, $\exists \varepsilon > 0:$ $z$ lies in the line $x =
\frac{1}{2} +\varepsilon$. So the image under $w$ lies in the circle,
\[u =
  \left(\frac{1}{2}+\varepsilon\right)(u^2+v^2) =
  \frac{1+\varepsilon}{2}(u^2+v^2)\implies u^2+v^2 =
  \frac{2}{1+\varepsilon}u \implies u^2-\frac{2}{1+\varepsilon}u +v^2
  = 0 \]
\[\implies u^2-\frac{2}{1+\varepsilon}u + \frac{1}{(1+\varepsilon)^2} +v^2
  = \frac{1}{(1+\varepsilon)^2} \implies  \left(u-\frac{1}{1+\varepsilon}\right)^2 +v^2
  = \frac{1}{(1+\varepsilon)^2}\]
Which is the circle of radius $\frac{1}{1+\varepsilon}$ centered
at $\frac{1}{1+\varepsilon}.$ That is the circle
$|w-\frac{1}{1+\varepsilon}| = \frac{1}{1+\varepsilon}$.
The circle $|w -\frac{1}{1+\varepsilon}|=
\frac{1}{1+\varepsilon}$ has the following parametric representation,
\[ w = \frac{1}{1+\varepsilon} + \frac{1}{1+\varepsilon} \exp
  i\theta \quad(0\leq \theta < 2\pi)\]

We want to show that the distance from $\frac{1}{1+\varepsilon} + \frac{1}{1+\varepsilon} \exp
(i\theta)$ to $1$ is less than  or equal to $1$.
\[\bigg|\frac{1}{1+\varepsilon} + \frac{1}{1+\varepsilon} \exp
  (i\theta) -1 \bigg| = \bigg|\frac{1}{1+\varepsilon} -1 +
  \frac{1}{1+\varepsilon} (\cos \theta +i\sin \theta)\bigg| = \bigg|\frac{1}{1+\varepsilon} -1 +
  \frac{1}{1+\varepsilon} \cos \theta +i\frac{1}{1+\varepsilon} \sin
  \theta\,\bigg| \]
\[= \bigg|\frac{1}{1+\varepsilon}(1+\cos \theta) -1
  +i\frac{1}{1+\varepsilon} \sin \theta\bigg| = \sqrt{\left(\frac{1}{1+\varepsilon}(1+\cos \theta) -1\right)^2
  +\frac{1}{(1+\varepsilon)^2}\sin^2\theta}\]
\[= \sqrt{\frac{1}{(1+\varepsilon)^2}(1+\cos \theta)^2 -2\frac{1}{1+\varepsilon}(1+\cos \theta)+1
    +\frac{1}{(1+\varepsilon)^2}\sin^2\theta}\]
\[= \sqrt{\frac{1}{(1+\varepsilon)^2}(1+2\cos \theta + {\color{orange}
      \cos^2\theta}) -2\frac{1}{1+\varepsilon}(1+\cos \theta)+1
    +{\color{orange}\frac{1}{(1+\varepsilon)^2}\sin^2\theta}}\]
\[= \sqrt{{\color{orange}\frac{1}{(1+\varepsilon)^2}} + \frac{1}{(1+\varepsilon)^2}(1+2\cos \theta) -2\frac{1}{1+\varepsilon}(1+\cos \theta)+1}\]
\[= \sqrt{\frac{1}{(1+\varepsilon)^2} +
    \frac{1}{(1+\varepsilon)^2}+\frac{2}{(1+\varepsilon)^2}\cos \theta
    -\frac{2}{1+\varepsilon} -\frac{2}{1+\varepsilon}\cos \theta +1}\]
\[= \sqrt{\frac{2}{(1+\varepsilon)^2} +\left(\frac{2}{(1+\varepsilon)^2} -\frac{2}{1+\varepsilon}\right)\cos \theta
    +1-\frac{2}{1+\varepsilon}}\]
\[= \frac{\sqrt{2 +\left(2 -2(1+\varepsilon)\right)\cos \theta
    +(1+\varepsilon)^2-2(1+\varepsilon)}}{1+\varepsilon}= \frac{\sqrt{2 +\left(2 -2 -2\varepsilon\right)\cos \theta
    +1+2\varepsilon + \varepsilon^2-2-2\varepsilon}}{1+\varepsilon}\]
\[= \frac{\sqrt{\left(-2\varepsilon\right)\cos \theta
      +1+ \varepsilon^2}}{1+\varepsilon} = \frac{\sqrt{
      1 -2\varepsilon \cos \theta + \varepsilon^2 }}{1+\varepsilon}\]

Since $-1<\cos \theta < 1 $ when $0\leq \theta <2\pi$, it follows
that,\[ 0 < \frac{\sqrt{
       1 -2\varepsilon \cos \theta +\varepsilon^2 }}{1+\varepsilon}\leq \frac{\sqrt{
      \varepsilon^2 +2\varepsilon +1 }}{1+\varepsilon} =
  \frac{\sqrt{(1+\varepsilon)^2}}{1+\varepsilon} =
    \frac{1+\varepsilon}{1+\varepsilon} = 1\]

The distance from the centers is $1$ only when $\theta = \pi$. So, $0$ is a
common point of the circles. Otherwise, the distance is less than $1$.

So the circle $\big|w-\frac{1}{1+\varepsilon}|=\frac{1}{1+\varepsilon}$
lies inside the circle $|w-1|=1$. Since every $w(z):$ Re $z$, lies in the
image of one of such lines. Then the right half plane Re$(z) >\frac{1}{2}$ lies inside
the disk $|w-1|<1\quad \blacksquare$
\newpage
\subsubsection*{Extra Credit Problem } Show that all four zeros of the polynomial $g(z) = z^4 - 7z - 1$ lie in the disk $|z| < 2$

\uwave{slu.}
Let $C$ be the circle $|z|=2$. Let $p(z) = z^4$, $q(z)=
-7z-1$. Notice, $g = p + q $ then if $z \in C$,
\[|p(z)|< |z|^4 = 2^4 =16 \text{ and }|q(z)| < 7|z|+1 = 2\cdot 7 +1 = 15\]
\[\implies \forall z\in \C, |q(z)|<|p(z)|\]
Both $p(z)$, and $q(z)$ are polynomials, so they're entire. So they're
analytic in the closed disk $|z|leq 2$. Therefore by Rouché's theorem,
$p(z)$ and $g(z)$ have the same number of zeros counting
multiplicities inside $C$. So, all four zeros of $g$ lie in the disk
$|z|<2\quad \lozenge$

\subsubsection*{Extra Credit {\color{orange} STAR PROBLEM}} Show that
the parabola $2x = 1 − y^2$ is mapped onto the cardioid $ρ = 1 + \cos
\phi$ by the reciprocal transformation $w = \frac{1}{z}$.

\uwave{slu.}

Let $z\in \C: \exists x,y \in \R: z = x+iy$, then
\[x = \frac{z+\bar{z}}{2}\text{ and }y = \frac{z-\bar{z}}{2i}\]
\[\implies y^2 = \left(\frac{z-\bar{z}}{2i}\right)^2 =
  \frac{z^2-2z\bar{z}+\bar{z}^2}{-4}\]
\[\implies -y^2 = -\left(\frac{z-\bar{z}}{2i}\right)^2\]
\[2x = 1 − y^2 \iff 2\frac{z+\bar{z}}{2} = 1 -\left(\frac{z-\bar{z}}{2i}\right)^2 \]
Let $z = \rho e^{i\phi},\,\, (0\leq \phi < 2\pi)\implies \bar{z} =
\rho e^{-i\phi}$ and plug in,
\[\implies 2\frac{\rho e^{i\phi}+ \rho e^{-i\phi}}{2} = 1 -\left(\frac{\rho
      e^{i\phi}-\rho e^{-i\phi}}{2i}\right)^2 \implies 2\rho\frac{e^{i\phi}+e^{-i\phi}}{2} = 1 -\rho^2\left(\frac{
      e^{i\phi}- e^{-i\phi}}{2i}\right)^2\]
\[\implies 2\rho \cos \phi = 1 -\rho^2\sin^2\phi \implies 2\rho \cos
  \phi = 1 -\rho^2(1-\cos^2 \phi)\]
\[\implies \rho^2(1-\cos^2 \phi) +2\rho\cos
  \phi = 1 \implies \rho^2 +2\rho\cos
  \phi -\rho^2\cos^2 \phi  = 1 \]
\[\implies (\rho-\cos\phi)^2 = 1 \implies \rho-\cos \phi = 1 \]\
\[\implies\rho = 1 + \cos \phi\quad \blacksquare\]
\newpage
\subsubsection*{Appendix}

I misread the exam and solved the following problem, I'll leave it
here for your amusement.

Consider the integral \[\int_C
  \frac{2z^3+3}{(z+1)(z^{\color{cyan} 4} +4)} dz\] taken counterclockwise around the
curve $C$

(a) Find the value of the integral when the curve $C$ is the circle
${\color{green}|z + 1| = 2}$

\uwave{slu.}

\[(z+1)(z^4+4) = 0\implies z+1 = 0 \text{ or } z^4+4 = 0\]
\[z+1 = 0 \implies z_0 = -1\]
\[z^4+4 = 0 \implies z^4 = -4 = -1\cdot 4 = 4 e^{i\left(  \pi+
      2k\pi\right)},\,\, k\in \Z \implies z =
  \sqrt{2}e^{i\left(  \frac{\pi}{4}+ \frac{k\pi}{2} \right)}= \pm(1\pm
  i)\]
\[z^4+4 = 0 \implies z_1 = 1+i,z_2 = -1+i, z_3 = -1-i, z_4 = 1-i\]
\[z_0 = -1 \implies |-1+1| = 0 <2 \implies z_0\in C\]
\[z_1 = 1+i \implies |1+i+1| = |2+i| = \sqrt{5}\]\[ 4<5 \implies
  \sqrt{4} = 2< \sqrt{5}  \implies z_1\not\in C\]
\[z_2 = -1+i \implies |-1+i+1| = |i| = 1 < 2\implies z_2 \in C\]
\[z_3 = -1-i \implies |-1-i+1| = |-i| = 1 < 2\implies  z_3 \in C\]
\[z_4 = 1-i \implies |1-i+1| = |2-i| = \sqrt{5} > 2\implies z_4
  \not\in C\]
Note, that finding all the zeros of $z^4+4$ gives us the following factorization,
\[z^4+4 = (z-(1+i))(z-(-1+i))(z-(-1-i))(z-(1-i)) \]
\[\text{Then } f(z) = \frac{2z^3+3}{(z+1)(z^4 +4)} = \frac{2z^3+3}{(z+1)(z-(1+i))(z-(-1+i))(z-(-1-i))(z-(1-i))}\]
\begin{align*}\implies \res_{z=-1} f(z) &=
  \frac{2(-1)^3+3}{(-1-(1+i))(-1-(-1+i))(-1-(-1-i))(-1-(1-i))}\\ &=
  \frac{1}{(-2-i)(-i)(i)(-2+i)}=  \frac{1}{((-2)^2-i^2)} \\&= \frac{1}{5}\end{align*}
\begin{align*}\implies \res_{z=-1+i} f(z) &=
                                       \frac{2(-1+i)^3+3}{(-1+i+1)(-1+i-(1+i))(-1+i-(-1-i))(-1+i-(1-i))}\\
                                     &=\frac{2(-1+i)^3+3}{i(-2)(2i)(-2+2i)}
                                     =\frac{2(-1+i)(-2i)+3}{i(-2)(2i)(-2+2i)}
                                     =\frac{2(2i+2)+3}{i(-2)(2i)(-2+2i)}\\
                                     &=\frac{7+4i}{i(-2)(2i)(-2+2i)}
                                     =\frac{7+4i}{4(-2+2i)}
                                       =\frac{7+4i}{-8+8i}
  = \frac{(7+4i)(-8-8i)}{128}\\  &=  \frac{-56 + 32 + (-32-56)i}{128} =
    \frac{-24 -88i}{128} = -\frac{24}{128} -\frac{88}{128}i \\
  &= -\frac{3}{16} - \frac{11}{16}i
\end{align*}
\begin{align*}\implies \res_{z=-1-i} f(z)
  &= \frac{2(-1-i)^3+3}{(-1-i+1)(-1-i-(1+i))(-1-i-(-1+i))(-1-i-(1-i))}\\
  &=\frac{2(-1-i)^3+3}{-i(-2-2i)(-2i)(-2)}
    =\frac{2(-1-i)^3+3}{i(2+2i)(2i)(2)}
    =\frac{2(-1-i)^3+3}{-4(2+2i)}
    =\frac{2(-1-i)^3+3}{-8-8i}\\
  &=\frac{2(-1-i)^3+3}{8(-1-1i)} =\frac{(-1-i)^2}{4}
    +\frac{3}{8(-1-i)} = \frac{2i}{4}
    +\frac{3(-1+i)}{16}  = \frac{i}{2}
    +\frac{3(-1+i)}{16} = -\frac{3}{16}+ \frac{11}{16}i
\end{align*}

So Cauchy's residue theorem gives us that,
 \begin{align*}\int_C
  \frac{2z^3+3}{(z+1)(z^4 +4)} dz &= 2\pi i\left( \frac{1}{5}
    -\frac{3}{16} - \frac{11}{16}i -\frac{3}{16} + \frac{11}{16}i \right)\\&= 2\pi i\left( \frac{1}{5}
    -\frac{6}{16} \right) = 2\pi i\left( \frac{16-30}{80} \right)\\ &=
  \pi i\left( \frac{-14}{40} \right)\\ &= -\frac{7\pi}{20}i\end{align*}

(b) find the value of the integral when the curve $C$ is the circle
$|z|=4$

Since, $|\pm(1\pm i)| = \sqrt{2} < 4$, and $|-1|<4$, we need to
compute a few more residues.

\begin{align*}\implies \res_{z=1+i} f(z)
  &=  \frac{2(1+i)^3+3}{(1+i+1)(1+i-(-1+i))(1+i-(-1-i))(1+i-(1-i))}\\
  &=  \frac{2(1+i)^3+3}{(2+i)(2)(2+2i)(2i)} =
    \frac{(1+i)^2}{4i(2+i)}-\frac{3i}{8(2+i)(1+i)} =
    \frac{2i}{4i(2+i)}-\frac{3i}{8(1+3i)}\\&=
    \frac{2-i}{10}-\frac{3i(1-3i)}{80}
  =  \frac{2-i}{10}-\frac{9+3i}{80} =  \frac{16-8i-(9+3i)}{80}\\ &=
  \frac{7}{80}-\frac{11}{80}i
\end{align*}


\begin{align*}\implies \res_{z=1-i} f(z)
  &=  \frac{2(1-i)^3+3}{(1-i+1)(1-i-(-1+i))(1-i-(-1-i))(1-i-(1+i))}\\
  &=  \frac{2(1-i)^3+3}{(2-i)(2-2i)(2)(-2i)}=
    \frac{2(1-i)^3+3}{(2-i)(1-i)(-8i)} = \frac{-2i}{(2-i)(-4i)} +
    \frac{3i}{8(2-i)(1-i)}\\
    &= \frac{1}{2(2-i)} + \frac{3i}{8(1-3i)} = \frac{2+i}{10} +
      \frac{3i(1+3i)}{80} = \frac{2+i}{10}
      -\frac{9-3i}{80} = \frac{16+8i - 9+3i}{80}\\ &= \frac{7}{80} +\frac{11}{80}i
\end{align*}
\begin{sagesilent}
  var('z', domain = 'complex')
  f = (2*z^3+3)/((z+1)*(z^4+4))
  s = f.residue(z==-1)
  s +=f.residue(z== -1+I)
  s += f.residue(z== -1-I)
  s += f.residue(z== 1-I)
  s += f.residue(z== 1+I)
\end{sagesilent}
So Cauchy's residue theorem gives us that,
 \begin{align*}\int_C
  \frac{2z^3+3}{(z+1)(z^4 +4)} dz &= 2\pi i\left( \frac{16-30}{80}
                                    +\frac{7}{80}-\frac{11}{80}i +
                                    \frac{7}{80}+\frac{11}{80}i
                                    \right)\\ &=
                                                2\pi i\left(
                                                \frac{16+14-30}{80}
                                                \right) = \sage{2*pi*I*s} \quad
                                                \blacksquare
 \end{align*}


 (c) Give a curve $C$ such that the value of the integral is $0$

 \uwave{slu. } Let $C$ be the circle $|z| = 4\quad \lozenge$

\end{document}
%%% Local Variables:
%%% mode: latex
%%% TeX-master: t
%%% End:
